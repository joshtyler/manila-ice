\documentclass{standalone}


\usepackage{bytefield}

% Memory layout taken from https://www.martin-demling.de/2011/06/memory-maps-in-latex-using-the-bytefield-package/
\newcommand{\memsection}[4]{
	\bytefieldsetup{bitheight=#3\baselineskip}    % define the height of the memsection
	\bitbox[]{10}{
		\texttt{#1}     % print end address
		\\ \vspace{#3\baselineskip} \vspace{-2\baselineskip} \vspace{-#3pt} % do some spacing
		\texttt{#2} % print start address
	}
	\bitbox{16}{#4} % print box with caption
}

\begin{document}
\begin{bytefield}{24}
	\memsection{ffffff}{100000}{10}{Free}\\
	\begin{rightwordgroup}{Slot 3}
		\memsection{0fffff}{0e0fbc}{3}{Application specific data}\\
		\memsection{0e0fbb}{0c0000}{4}{Image}
	\end{rightwordgroup}\\
	\begin{rightwordgroup}{Slot 2}
		\memsection{0bffff}{0a0fbc}{3}{Application specific data}\\
		\memsection{0a0fbb}{080000}{4}{Image}
	\end{rightwordgroup}\\
	\begin{rightwordgroup}{Slot 1}
		\memsection{07ffff}{060fbc}{3}{Application specific data}\\
		\memsection{060fbb}{040000}{4}{Image}
	\end{rightwordgroup}\\
	\begin{rightwordgroup}{Slot 0 (Bootloader) \\ \textbf{Write protected}}
		\memsection{03ffff}{02105c}{3}{Reserved}\\
		\memsection{02105b}{0000a0}{4}{Bootloader image}\\
		\memsection{00009f}{000000}{3}{Multiboot header}
	\end{rightwordgroup}\\
\end{bytefield}
\end{document}